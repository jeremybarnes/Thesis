% intro.tex
% Jeremy Barnes, 21/9/1999
% $Id$

\chapter{Introduction}

The project undertaken was an investigation of a method of
overcoming a known problem (\emph{overfitting}) of a particular
machine learning algorithm (the \emph{Boosting} algorithm).  The
result is a series of machine learning algorithms called
\emph{$p$-boosting algorithms}.  These algorithms are developed,
analysed and tested in later chapters.


\section{Overview}

This introduction provides an overview of the rest of the thesis, and
covers the bigger picture, describing the field of machine learning
and its applications to practical problems.  The rest of the thesis
may is composed of three four parts.

\subsection*{Part I: Background}

The next three chapters rapidly narrow the focus.  Chapter
\ref{chapter:slt} describes the field of \emph{Statistical learning
theory} (SLT)\footnote{A table of acronyms is provided in the
preface.}, which provides much of the theoretical
foundations of (but by no means encompasses) the field of machine
learning.  Assumptions of \emph{binary problems} and
\emph{classification problems} further restrict the scope of the
investigation.

Chapter \ref{chapter:stumps} provides a description of, and proves
some properties of, a very simple learning algorithm called
\emph{decision stumps} which fits within the restricted scope imposed
by the first two chapters.  This learning algorithm
is used as a basis for the more powerful Boosting algorithms described
later.

With the necessary background in place, the \emph{Boosting} algorithm
itself is the subject of chapter \ref{chapter:boosting}.  This
algorithm forms the starting point for the original work that is
considered in further chapters.  We consider what it means to
``boost'', adopting a broad definition, and many properties of the
Boosting algorithm are given.  Finally the \emph{problem} of overfitting in
boosting is investigatied, as a prelude to the \emph{solutions}
investigated in the second part of the thesis.


\subsection*{Part 2: Theoretical investigation}

The key inventive step of the project thesis is the use of a
\emph{$p$-convex hull} to reduce the problem of overfitting.  Chapter 
\ref{chapter:pboosting} investigates the theoretical justification
behind this line of reasoning in some detail.  These abstract ideas
are then developed into concrete algorithms, and properties of these
algorithms are considered.

Chapter \ref{chapter:method} describes how these algorithms were
tested.  Chapter \ref{chapter:results} details the main results of the
project (more detail is available in the appendixes and the complete
set of results on the attached CD-ROM inside the back cover).

\subsection*{Part 3: Evaluation}

The thesis concludes with a discussion (chapter
\ref{chapter:discussion}) and a conclusion (chapter
\ref{chapter:conclusion}.  These two sections evaulate the results
obtained in the context of the project and in the broader scope of the
field of machine learning and suggest avenues of further enquiry which
may be fruitful.

\section{The scheme of things}

This section provides an overview of the context in which this project
exists, starting very broadly and rapidly focusing on the immediate
background.


\subsection{Artificial intelligence}

The broadest possible subject area in which this work is contained is
often described as \emph{artificial intelligence}.  This field
essentially contains our efforts to make computers act in a similar
manner to humans.

This field encompasses a large body of history, philosophy and
knowledge (see, for example \cite{Penrose89}).  We will ignore much
of this, and concentrate on efforts to make a computer ``think'' like
a human.

Early work in this field (from the 1950s to the 1980s) was based on
the idea that intelligence can be emulated with a large set of rules.
The systems generated as a result, known as \emph{expert systems},
were huge databases of human-generated rules that were meant to
represent the complete knowledge of a human expert on a particular
domain.  These systems were reasonably successful at first, but became
unwieldy as the number of rules increased (a large amount of effort is
required to generate even the 1000 rules that were used in large
expert systems in 1980). 

The problem was more practical than theoretical: systems with large
numbers of rules can theoretically learn any learnable problem to a
desired accuracy; it is \emph{generating} the rules that is hard.  The
next step, obvious in hindsight, was to invent machines that could
generate their own rules; machines that could \emph{learn}.


\subsection{Machine learning}

Learning machines came in two flavours.  \emph{Supervised learners} are
shown examples of some kind of relationship, along with the ``right''
answers (generated from some form of supervisor) and attempt to learn a
relationship such that their answer always matches that of the
supervisor.  An example is trying to learn the likely outcome of a
court case based upon details of similar cases (the ``supervisor''
here is the information on the outcomes of cases on file).  All
algorithms discussed in this thesis are supervised algorithms.
\footnote{\emph{Unsupervised algorithms} are given a set of data and
asked to learn a pattern in the data (for example, identifying
interesting clumps of stars from telescope images).}


\subsection{Statistical learning theory}

Statistical learning theory provides the theoretical foundation upon
which machine learning rests.  It provides a body of knowledge that
allows bounds on the performance of learning algorithms to be
generated.  Much of this theory can be attributed to V.M.Vapnik
\cite{Vapnik98}.  Results from statistical learning theory enable us
to design learning machines which we can be sure will learn a function
that is close to the optimal.

\subsection{Voting methods}

\todo{Explain hypothesis vs algorithm (one sentence).}
Boosting and $p$-boosting are both examples of \emph{voting methods},
a relatively new class of algorithm which operates in a bootstrap-like
manner by combining many ``weak''\footnote{A definition of a weak
algorithm appears in chapter \ref{chapter:boosting}; as a rough guide any
algorithm that is not a combination of other algorithms may be
considered weak.} hypotheses to generate a composite hypothesis.  The
success of these algorithms has been remarkable; as a result of these
algorithms the focus of machine learning research in recent years to
shifted away from the classical algorithms (all of which are more or
less equal when boosted) and on to developments in voting methods.

These algorithms were initially observed to be immune to many of the
problems of overfitting.  Recent results have indicated that this is
unfortunately not the case.

\subsection{A real-world application}
\label{sec:churn example}

Consider a telephone company that is interested in predicting whether
its customers are likely to ``churn'' (change telephone companies) in
the near future.  Such a telephone company may have a database with
the data shown in table \ref{table:churn attributes} for several
customers; this data can be used to train a learning machine.  
The resultant learning machine can then be used to predict whether
current customers are likely to churn (classify customers into
``churn'' and ``not churn'' categories).  The company can then target
these customers with special offers or improved service.
This particular application will be revisited throughout the thesis.

\begin{table}
\newcommand{\szo}{\{0,1\}}
\begin{center}
\begin{tabular}{r l c l}
\small
{\bf No.} & {\bf Name} & {\bf Range} & {\bf Description} \\
\hline
 1 & length      & $\bbR$ & Weeks customer has been with telco \\
 2 & area        & $\bbN$ & Telephone area code \\
 3 & intnlp      & $\szo$ & Whether customer has international plan \\
 4 & mailp       & $\szo$ & Whether customer has mail plan \\
 5 & messages    & $\bbN$ & Number of voice mail messages \\
 6 & daymin      & $\bbR$ & Number of day minutes \\
 7 & daycalls    & $\bbN$ & Number of day calls \\
 8 & daycost     & $\bbR$ & Total cost of day calls \\
 9 & evemin      & $\bbR$ & Number of evening minutes \\
10 & evecalls    & $\bbN$ & Number of evening calls \\
$\vdots$ & $\vdots$ & $\vdots$ & $\vdots$ \\
19 & churn       & $\szo$ & {\bf Whether customer ``churned''} \\
\end{tabular}
\end{center}
\caption{Attributes of a the ``churn'' machine learning problem}
\label{table:churn attributes}
\end{table}

\section{Issues in machine learning}

Machine learning is a hard problem, and there are many issues
involved.  A real-world example of many of these problems concerns
efforts by the US army to detect tanks from photographs using a
learning machine.  The training data was two sets of photographs of
terrain; one set including tanks and the other without.  These
photographs were taken at different times during the day.  When the
learning machine was trained, it learned to detect differences in the
sun position rather than the presence of tanks! Although the learning
machine could classify the training data correctly, its generalisation
ability was poor.

This example illustrates the 
Overfitting, simply put, is learning the specifics of a process too
well to be able to apply the knowledge to a more general problem.
A familiar example for many people would be driving a well-known route
on ``auto-pilot'' and almost coming to grief on a new obstacle.  In
the context of machine learning, overfitting occurs when the learning
machine adapts to peculiarities or noise in the input data, to the
detriment of generalisation ability.

Statistical learning theory gives us a theoretical reason for
overfitting, and also an insight into how it may be avoided.  In
particular, by limiting the complexity of the \emph{set} of hypotheses
that the learning machine may generate we can avoid overfitting.


\section{Original work}

The original work of this project is motivated by observations of a
theoretical bound on overfitting; and in particular on a method of
reducing overfitting in the boosting algorithm.  This theory is first
developed into several practical algorithms; these algorithms are then
tested against the original AdaBoost algorithm and for compliance with
the theory behind them.





