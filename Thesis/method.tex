% method.tex
% Jeremy Barnes, 22/9/1999
% $Id$

% Method baby...

\chapter{Experiments}
\label{chapter:method}

Extensive simulations of the $p$-boosting algorithms and AdaBoost were
run on a microcomputer in order to compare their performance, both
with the baseline AdaBoost algorithm, and with the theory developed in
previous chapters.  This chapter describes the experimental setup
(including a brief description of the software developed) and the
testing methodology to a level of detail sufficient to repeat the
experiments.

\section{Experimental setup}

In this section a summary of the equipment and computer software used
to perform the experiments is given.  The full source code, datasets
and test results are available on the CD-ROM attached inside the back
cover of this thesis.  Appendix \ref{appendix:cdrom} describes the
contents of this CD-ROM in more detail; appendix
\ref{appendix:datasets} the datasets; and appendix
\ref{appendix:software} the computer software.

\subsection{Hardware}

The simulations were run on three microcomputers.  The first was a
400MHz Celeron system with 256MB of memory, running RedHat Linux
version 6.0 and \MATLAB\ version 5.3.0.10183 (Release 11).  The second
was a 300MHz Celeron system with 128MB of memory, running RedHat Linux
version 5.2 and the same version of \MATLAB.  The third was a 233MHz
Cyrix M3 machine with 64MB of memory running Windows 95 version
4.00.1111 and \MATLAB\ version 5.0.0.4073 (Student version).

\subsection{Software}

Although an incidental part of the project, the software package that
was developed in order to perform the simulations is a significant
piece of work in its own right.  It was written from scratch to
allow an accessible and efficient entry into the project area--which
was not available at the commencement of the project.  In particular,
it includes many functions that allow visualisation of the algorithms.
Use of this package will significantly lower the difficulty of
beginning similar research or extending the ideas developed in this
project.

The software was written as a \MATLAB\ toolbox.  Two weak learning
algorithms (decision stumps and CART), several versions of the
Boosting algorithm (including all mentioned in this thesis), an
implementation of a Neural Network algorithm, an automated test
harness, and assorted analysis and visualisation functions are
included.  What follows is a brief description of the design and
implementation of the software package as a whole.  Individual
components are covered in appendix \ref{appendix:software}.

\subsection{Optimisation and numerical issues}

The code was profiled extensively using \MATLAB's inbuilt profiler,
and efficient algorithms selected for the frequently executed
sections.  Tight sections of code, and certain data and code
structures (for example, {\tt for} loops), which are inefficient under
\MATLAB, were recoded in \C\ to improve speed.  (Details of how
to interface this code were obtained from \cite{MathWorks96} and
\cite{MathWorks96a}.) As each function was optimised, its output was
compared with that of the un-optimised version to ensure equivalence
of the two versions of code.  The net effect of these optimisations
was that simulations which would have required several years to run as
initially coded were able to be run in a matter of days.

Minimising numerical errors was a major design goal.  Several sections
of code incorporate safeguards (such as periodic full recalculations
in loops that are optimised incremental calculations) to minimise the
effect of numerical errors.  In addition, 64 bit IEEE double precision
floating point numbers are used exclusively.

In total, the source code was 270KB in size, comprising 235KB of
\MATLAB\ code and 35KB of \C code.  There are 10000 lines in 191
\MATLAB\ {\tt .m} files and 1500 lines in 6 \C\ source files.  All
source code was maintained under the {\tt CVS} version control
system.

\section{Datasets}

A total of seven datasets were used in the experiments.  Four
(\ds{ring0}, \ds{ring10}, \ds{ring20} and \ds{ring30}) were
synthetic datasets, randomly generated from a known distribution with
noise added artificially.  Two other datasets (\ds{sonar} and
\ds{wpbc}%
\footnote{Wisconsin Prognostic Breast Cancer.}
) were obtained from the UCI repository \cite{UCI}, and the
final dataset (\ds{acacia}) was ecological data used in a PhD thesis
\cite{Payne97}.  A summary of the features of each dataset appears in table
\ref{tbl:datasets} (Appendix \ref{appendix:datasets} describes the
datasets in more detail.)  This group was selected to cover a broad
range of situations:  the four \ds{ring} datasets allow the effect of
noise to be isolated and contain a very high sample-to-attribute
ratio;  the \ds{sonar} dataset is low-noise (generated from precise
measurements ) but contains a low sample-to-attribute ratio; and the
\ds{wpbc} and \ds{acacia} datasets are examples of difficult and very
difficult%
\footnote{Difficult and very difficult in that previous applications
of machine learning algorithms to these datasets produce indifferent
to poor results.}%
real-world data.

\begin{table}
\begin{center}
\begin{tabular}{l c c r r r r}\hline
{\bf Dataset} & $\calI$ & $\calO$ & {\bf Artificial noise} & {\bf
Size} & {\bf Training samples} & {\bf Test samples} \\
\hline \hline
\ds{ring0} & $[0,1]^2$ & $\{\pm 1\}$ & 0\% & $\infty$ & 50 & 5000 \\
\ds{ring10} & $[0,1]^2$ & $\{\pm 1\}$ & 10\% & $\infty$ & 50 & 5000 \\
\ds{ring20} & $[0,1]^2$ & $\{\pm 1\}$ & 20\% & $\infty$ & 50 & 5000 \\
\ds{ring30} & $[0,1]^2$ & $\{\pm 1\}$ & 30\% & $\infty$ & 50 & 5000 \\
\hline
\ds{sonar} & $[0,1]^{60}$ & $\{\pm 1\}$ & 0\% & 208 & 70 & 138 \\
\ds{wpbc} & $\subset \bbR^{32}$ & $\{\pm 1\}$ & 0\% & 194 & 97 & 97 \\
\ds{acacia} & $\subset \bbR^{16}$ & $\{\pm 1\}$ & 0\% & 204 & 102 & 102 \\
\hline
\end{tabular}

{\small Note that samples with missing attribute values were excluded
from each dataset.}
\end{center}
\caption{Summary of dataset attributes}
\label{tbl:datasets}
\end{table}

\section{Testing}

In this section we detail the aims of our experiments, the procedure
that was followed, and how the results were summarised and analysed.

\subsection{Aims}
The aims of the experiments were four-fold:
%
\begin{enumerate}
\item	To test the generalisation performance of the $p$-boosting
	algorithms, and how this performance compares with the
	that of AdaBoost.  It was expected that the $p$-boosting
	algorithms would outperform AdaBoost on noisy datasets.

\item	To verify that the qualitative behaviour (and quantitative
	behaviour wherever possible) of the $p$-boosting algorithms
	matched the theory developed in chapters \ref{chapter:slt},
	\ref{chapter:boosting} and \ref{chapter:pboosting}:
	%	
	\begin{itemize}
	\item	The $p$ parameter controls the capacity of the
		function class.  Thus, a generalisation error vs $p$
		plot should have a minimum at some optimal value
		$p^{\ast}$ (figure \ref{fig:optimal p value}).
	\item	The confidence interval in theorem \ref{thm:p convex
		generalisation} decreases as $p \rightarrow 0$.  As a
		result, the true risk (test error) and empirical risk 
		(training error) curves should match closely as $p
		\rightarrow 0$.
	\end{itemize}
	
\item	To verify that the algorithms were operating as designed.
	It was expected that a distribution of classifier weights
	would show that most of the weight is given to fewer and fewer
	classifiers for low $p$ values.

\item	To test the efficiency of the algorithms as compared to
	AdaBoost.  An algorithm that generates a slightly better
	hypothesis but takes much longer to train may not be
	considered an improvement in some applications.
\end{enumerate}

\subsection{Method}

A total of 31 tests were run, as detailed in table
\ref{tbl:experiments}.  Each test involved training a particular
algorithm on a dataset (up to a maximum of \emph{Iterations} training
iterations).  The test was repeated for \emph{Trials} independent
trials, each time with a newly generated (\ds{ring*}) or randomly
permuted and partitioned (\ds{sonar}, \ds{wpbc}, \ds{acacia})
test and training dataset%
\footnote{Note that noise was added to the training \ds{ring*}
datasets but \emph{not} the test \ds{ring*} datasets.}%
, and this whole procedure repeated for each $p$ value.

\newcommand{\allp}{$\frac{1}{2} \leq p \leq 2; 10p \in \bbN$}
\newcommand{\lowp}{$\frac{1}{2} \leq p \leq 1; 10p \in \bbN$}
\begin{table}
\begin{center}
\begin{tabular}{r l l r r c}\hline
{\bf Number} & {\bf Algorithm} & {\bf Dataset} & {\bf Trials} &
{\bf Iterations} & {\bf $p$ values} \\
\hline\hline
1-4  & AdaBoost & \ds{ring*}   & 100 &  1000 & - \\
5    & AdaBoost & \ds{sonar}   &  30 &  1000 & - \\
6    & AdaBoost & \ds{wpbc}    &  50 & 10000 & - \\
7    & AdaBoost & \ds{acacia}  &  50 & 10000 & - \\
\hline
8-11 & Na\"{\i}ve & \ds{ring*} &  50 &  1000 & \allp \\
12   & Na\"{\i}ve & \ds{sonar} &  30 &  1000 & \allp \\
\hline
13-16 & Strict & \ds{ring*}    &  30 &  1000 & \allp \\
17    & Strict & \ds{sonar}    &  50 & 10000 & \allp \\
18    & Strict & \ds{wpbc}     &  20 &  5000 & \allp \\
19    & Strict & \ds{acacia}   &  20 & 10000 & \allp \\
\hline
21-24 & Sloppy & \ds{ring*}    &  30 &  1000 & \allp \\
25-28 & Sloppy & \ds{ring*}    &  30 & 10000 & \lowp \\
29    & Sloppy & \ds{sonar}    &  50 & 10000 & \allp \\
30    & Sloppy & \ds{wpbc}     &  20 &  5000 & \allp \\
31    & Sloppy & \ds{acacia}   &  20 & 10000 & \allp \\
\hline  
\end{tabular}

\small{The weak learning algorithm is always \emph{decision stumps}.}
\end{center}
\caption{Summary of experiments conducted}
\label{tbl:experiments}
\end{table}

The following data was recorded for each trial:
%
\begin{itemize}
\item	All test parameters;
\item	Test and training datasets;
\item	Training and test error (unweighted) at each iteration;
\item	Margins of the training samples at both the iteration with the
	minimum \emph{test} error and the final iteration;
\item	Classifier weights at the final iteration.
\end{itemize}


\subsection{Analysis}

More than two gigabytes of data was generated in the course of
testing.  This data was summarised across all trials for each (test,
p-value) combination, and a summary file created.  The summary file
contained the following information:
%
\begin{itemize}
\item	The mean and standard deviation of test and training error
	curves at each iteration;
\item	The number of trials that had not aborted before each
	iteration (recall that the algorithms would abort if the
	training error exceeded $\frac{1}{2}$, fell below $0$, or if
	the cost function was strictly increasing); 
\item	The value of the best (lowest) test error
	$\underset{X_T}{R_{\emp}}$ for each trial, and the number of
	training iterations after which this best error value
	occurred;
\end{itemize}
%
Each of the $p$-boosting algorithms had further statistics produced
for each $p$ value.  These statistics were:
%
\begin{itemize}
\item	The mean and standard deviation of the best test error at each
	$p$ value, over all trials;
\item	The mean and standard deviation of the number of training
	iterations the best error was produced at, over all trials.
\end{itemize}

For each of the $p$-boosting algorithms, a crude form of structural
risk minimisation (see section \ref{sec:theoretical overfitting}) was
then performed, by selecting the $p$ value with the \emph{lowest mean
best test error} to be $p^{\ast}$ (see figure \ref{fig:effect of p} on
page \pageref{fig:effect of p}).  This is the $p$ value that is used 
whenever comparisons of the performance of algorithms are made (it
makes little sense to compare anything but the \emph{best}
performance!)  The entire analysis described above was repeated for
each dataset.





