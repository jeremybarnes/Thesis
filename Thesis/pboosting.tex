% pboosting.tex
% Jeremy Barnes, 22/9/1999
% $Id$

\chapter{$p$-boosting}

This chapter describes the concepts behind and a theoretical
justification of \emph{$p$-boosting}, a generalisation of the boosting
algorithm.  The treatment is somewhat abstract; further chapters will
be more concrete as practical issues are considered.

\section{Motivation: avoiding overfitting}

\subsection{Qualitative arguments}

\subsection{Quantitative arguments}
* By reducing covering numbers we get a better bound
* Thus use a $p$-convex hull instead of a $1$-convex hull
* Trade off: minimum margin may be reduced
* Thus, we would expect to see a curve (draw a graph, shaped like a parabola)
* Optimal $p$ value there, gives us the best error
* Draw a picture of the classes being smaller

\subsection{Summary}

\section{Development of algorithms}

\subsection{Naive algorithm}

\subsection{Strict algorithm}

\subsection{Sloppy algorithm}

\subsection{Gravity algorithm}

\subsection{Summary}

\section{Chapter summary}



