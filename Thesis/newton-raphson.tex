% newton-raphson.tex
% Jeremy Barnes, 16/10/1999
% $Id$

\chapter{Details of Newton-Raphson method}
\label{appendix:newton-raphson}

This appendix details Newton-Raphson search implemented in the sloppy
and strict algorithms.

The following code sequence, which is edited to remove debugging code,
is obtained from the file {\tt trainagain.m} in the {\tt normboost}
object folder.

\small
\begin{verbatim}
function [obj_r, context] = trainagain(obj)

if (aborted(obj))
   obj_r = obj;
   warning('trainagain: attempt to train when training aborted');
   return;
end

x_data = x(obj);  y_data = y(obj);  w_data = w(obj);  p = norm(obj);

% create and train a new classifier
new_c = train(weaklearner(obj), x_data, y_data, w_data);

% find the training error
new_error = training_error(new_c);

% see what this algorithm does to our data
new_y = classify(new_c, x_data);


% find if we need to abort
if (new_error == 0)
   % Zero error -- one classifier can do perfectly by itself
   obj_r = abort(obj);
   return;
end

if (new_error >= 0.5)
   % Error of 0.5 -- random guessing does just as well
   obj_r = abort(obj);
   return;
end

% Calculate alpha
if (iterations(obj) == 0)
   % First iteration -- set alpha to 1 and marg to the margins of the
   % trained weak learner.

   alpha = 1.0;
   marg = (y_data*2-1) .* (new_y*2-1);
   new_b = [1.0];

else
   % Calculate alpha using a line search.  This uses the Newton-Raphson method
   % to find the minimum.

   % Initialisation
   new_alpha = 0.5 / iterations(obj);
   d = 1;
   min_d = inf;
   max_d = -inf;
   last_c = 1000000;
   tolerance = 0.001;
   
   
   % Iterate
   iter = 0;
   while ((abs(d) >= tolerance) & (iter < 1000))
      alpha = new_alpha;
      
      [c, d, d2] = eval_cf(obj, new_c, alpha);
      min_d = min([d min_d]);
      max_d = max([d max_d]);
      
      % Handle d2=0 in a very crude manner, which allows us to continue
      if (d2 < eps)
	 d2 = 0.01;
      end
      
      new_alpha = alpha - (d / d2);
      
      % Never let alpha move more than half the distance to zero.  This
      % can lead to alpha < zero, which causes all kinds of problems.
      min_alpha = alpha / 2;
      new_alpha = max([min_alpha new_alpha]);
      
      % This stopping rule is designed to detect the situation where there
      % is no solution as dc/dalpha > 0 for all alpha.  It halts the
      % training if there is no change in the cost function, and we have
      % not found a zero crossing of the derivative.
      if ((abs(c - last_c) < tolerance) & (min_d*max_d > 0))
	 break;
      end
      last_c = c;
      
      iter = iter + 1;
   end

   % Test for lack of convergence
   if (iter >= 100)
      warning('trainagain: failed to converge after 100 iterations');
   end
   
   % Test for minimum/maximum at our found point.  We do this on the sign
   % of the second derivative: if positive, we found a minimum; if
   % negative we found a maximum.
   [c, d, d2, marg] = eval_cf(obj, new_c, alpha);
   
   if (d2 <= 0.0)
      alpha = 0;
      obj = abort(obj);
   end
   
   % Calculate our new classifier weights   
   old_b = classifier_weights(obj);
   new_b = [old_b alpha] ./ pnorm([old_b alpha], p);

   one_norm = pnorm(new_b, 1)
end

% Update the sample weights.  These are calculated as the derivatives
% of the cost _function_ (not functional) and then normalised with a
% 1-norm equal to 1.
%
% For a cost function of exp(-x), dc/dx = -exp(-x).  The minus gets
% normalised out, and can be ignored.

new_w = exp(-marg);
new_w = new_w ./ sum(new_w);


% Update our weights
obj = add_iteration(obj, new_c, new_b, new_w);
obj.margins = marg;

obj_r = obj;
\end{verbatim}
