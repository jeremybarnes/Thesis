% commands.tex
% Jeremy Barnes, 1999
% $Id$

\providecommand{\emp}{\mathrm{emp}}
\providecommand{\calF}{\ensuremath{\mathcal{F}}}
\providecommand{\fat}{\ensuremath{\mathrm{fat}}}
\providecommand{\sign}{\ensuremath{\mathrm{sign}}}
\providecommand{\cop}{\ensuremath{\mathrm{co}_p}}
\providecommand{\co}{\ensuremath{\mathrm{co}}}
\providecommand{\bfx}{\ensuremath{\mathbf{x}}}
\providecommand{\bfy}{\ensuremath{\mathbf{y}}}
\providecommand{\bfyh}{\ensuremath{\hat{\mathbf{y}}}}
\providecommand{\calO}{\ensuremath{\mathcal{O}}}
\providecommand{\calI}{\ensuremath{\mathcal{I}}}
\providecommand{\calH}{\ensuremath{\mathcal{H}}}
\providecommand{\calX}{\ensuremath{\mathcal{X}}}
\providecommand{\ip}[2]{\ensuremath{\langle {#1} , {#2} \rangle}}
\providecommand{\lin}{\mathrm{lin}}
\providecommand{\calS}{\ensuremath{\mathcal{S}}}
\providecommand{\VCdim}{\mathrm{VCdim}}
\providecommand{\Fat}[1]{\mathrm{Fat}_{#1}}
\providecommand{\cover}[2]{\mathcal{N}({#1}, {#2})}
\providecommand{\covert}[3]{\mathcal{N}({#1}, {#2}, {#3})}
\providecommand{\MATLAB}{{\tt MATLAB}}
\providecommand{\C}{{\tt C}}
\providecommand{\argmin}{\mathrm{argmin}}

% Theorem-like constructs
\newtheorem{theorem}{Theorem}
\newtheorem{definition}{Definition}

\providecommand{\proof}{\par \par \noindent {\bf Proof:\ }}

\providecommand{\figlinewidth}{1pt}

\newenvironment{linefigure}%
		{\begin{figure} \rule{\textwidth}{\figlinewidth}}%
		{\rule{\textwidth}{\figlinewidth}\end{figure}}


\chapter{Statistical learning theory}
\label{chapter:slt}

This chapter explores the theoretical background behind the
boosting algorithm.  The main aim of the chapter is to develop a
theoretical framework that can be used in later chapters to generate
bounds on the performance of the $p$-boosting algorithm.

This chapter also serves as a literature review.  Several papers were
included into this chapter, and the relevant contributions are
included in this section.

The first section defines the ``machine learning'' problem and
the classification problem, including some definitions that are
followed throughout the rest of the thesis.  The pace is slow,
allowing a reader with little previous exposure to grasp the context
of the work.  It can probably be skipped by readers who are familiar
with this field.

The second section introduces the topic of statistical learning theory,
upon which the important theoretical results of this thesis are based.
The treatment is by necessity brief, and only the main results are
stated.

The third section introduces the two main classifiers that are used:
decision stumps and a modification of the CART algorithm.  The CART
algorithm is viewed as an extension of the decision stumps algorithm,
and is covered in little detail.

The fourth section defines and analyses the boosting algorithm, of
which $p$-boosting is a variant.  Several results are included from
previous work that has attempted to bound the generalisation error of
the boosting algorithm using the classical statistical learning theory
results.

The fifth section introduces the concept of the ``margin'', which has
been used by some authors to explain the performance of the boosting
algorithm.  Two related concepts, those of the ``fat-shattering
dimension'' and ``covering numbers'' are also discussed.

The chapter concludes with results from various papers that use the
fat-shattering dimension or covering numbers to generate bounds on the
performance of the boosting algorithm.  These bounds are compared with
those developed earlier using straight statistical learning theory.




\section{Formulation of machine learning}

Before we can begin describing the theorems that are a result of
statistical learning theory, we first need to develop some conventions
and notation.

\subsection{Learning methods}
A ``learning method'' is described in Cherkassky and Mulier
\cite{Cherkassky98} as
%
\begin{quote}
	\ldots an algorithm (usually implemented in software) that
	estimates an unknown mapping (dependency) between a system's
	inputs and outputs from the available data, namely from known
	(input, output) samples.
\end{quote}
%
This definition motivates the concept of \emph{supervised learning},
which is illustrated in figure \ref{fig:supervised learning}.  The generator
produces samples $\mathbf{x}$ drawn from a fixed (but unknown)
probability distribution $p(x)$.  These samples are presented both to
a supervisor (which produces the ``correct'' result $\mathbf{y}$) and to the
learning algorithm (which produces its estimate $\hat{\mathbf{y}}$).  The
learning algorithm can use the difference between these two results
($\bfyh - \bfy$) as feedback to improve its approximation.  The goal
is for the learning machine to always generate the correct
estimate, such that $\mathbf{\hat{y} - y = 0}$ for all samples $\mathbf{x}$.

\begin{linefigure}
\begin{center}
\begin{picture}(270,155)(30,20)
\put(30,110){\framebox(60, 60){\parbox{55pt}{\center{Sample generator}}}}
\put(150,110){\framebox(60, 60){\parbox{55pt}{\center{Learning machine
$\hat{h}(\mathbf{x})$}}}}
\put(150,30){\framebox(60, 40){\parbox{55pt}{\center{Supervisor
$h(\mathbf{x})$}}}}
\put(90, 140){\vector(1,0){60}}
\put(120,140){\line(0,-1){90}}
\put(120,50){\vector(1,0){30}}
\put(120,150){\framebox(0,0){$\mathbf{x}$}}
\put(210,50){\line(1,0){30}}
\put(240,130){\line(0,-1){80}}
\put(250,90){\framebox(0,0){$\mathbf{y}$}}
\put(240,130){\vector(-1,0){30}}
\put(210,150){\vector(1,0){60}}
\put(280,150){\framebox(0,0){$\hat{\mathbf{y}}$}}
\end{picture}
\end{center}
\caption{Supervised learning}
\label{fig:supervised learning}
\end{linefigure}

Each component of figure \ref{fig:supervised learning} will now be
described.


\subsubsection{Sample generator}
\label{sec:sample generator}

The sample generator drives the learning process, by generating a
series of samples $\mathbf{x}_1, \ldots$ for the learning machine to
operate on.  Each $\mathbf{x}_i \in \mathcal{I}$ where $\mathcal{I}$
is the ``input space'', and $\mathcal{I} \subseteq \mathbb{R}^n$ where
$n$ is the dimensionality of the sample space.

The input set depends upon number and range of input variables in the
problem; in the datasets considered in this thesis, usually
$\mathcal{I} = [0,1] \times [0,1] \subset \mathbb{R}^2$.


\subsubsection{Supervisor}
\label{sec:supervisor}

The supervisor is a function
%
\begin{equation}
\mathbf{y} = h(\mathbf{x}) \qquad \mbox{where} \qquad \mathbf{x} \in
\mathcal{I} \qquad \mbox{and} \qquad \mathbf{y} \in \mathcal{O}
\label{eqn:supervisor}
\end{equation}
%
that always generates the ``correct'' answer
$\mathbf{y}$\footnote{Correct \emph{answer} or drawn from correct
\emph{distribution}?} given an input $\mathbf{x}$.  The ``output set''
$\mathcal{O}$ is similar to the input set; it is defined as $\mathcal{I}
\subseteq \mathbb{R}^m$ where $m$ is the dimensionality of the output
space (or the number of output variables).

In almost all cases, there is only one output variable ($m=1$) or the
multiple output variables are independent and can each
be generated by a separate learning machine with $m=1$.
Thus it is reasonable to restrict our attention to the case where
$m=1$ for the rest of this thesis.

The distinction between \emph{classification} and \emph{regression}
problems is made on the size of $\mathcal{O}$.  This distinction is 
considered further in sections \ref{sec:regression} (regression) and
\ref{sec:classification} (classification).


\subsubsection{Learning machine}
\label{sec:learning machine}

The learning machine is a function with a form very similar to that of
the supervisor:
%
\begin{equation}
\mathbf{\hat{y}} = \hat{h}(\mathbf{x}) \qquad \mbox{where} \qquad
\mathbf{x} \in \mathcal{I} \qquad \mbox{and} \qquad \mathbf{y} \in
\mathcal{O}
\label{eqn:learning machine formulation}
\end{equation}
%
The major difference is that the learning machine also includes
learning behaviour (not shown in equation \ref{eqn:learning machine
formulation}, whereby the function $\hat{h}(\cdot)$ is updated
based upon the ``correct'' value $\mathbf{y}$ generated by the
supervisor (refer back to figure \ref{fig:supervised learning}.
\emph{The focus of machine learning theory is on generating learning
machines that operate as close as possible to the supervisor.}



\subsection{Regression}
\label{sec:regression}

The regression problem is defined as one where the size of the output
space $\|\mathcal{O}\|$ is infinite.  Usually, this means that
$\mathcal{O}$ is some interval on $\mathbb{R}$.  This thesis does not
consider regression problems (although it may not be difficult to
generalise some of the results to include regression problems); thus
regression is not considered further.




\subsection{Classification}
\label{sec:classification}

In the classification problem, the size $c = \| \mathcal{O} \|$ is
finite.  Thus, our permissible
$y$ values are now a finite number of discrete values (``categories)):
%
\begin{equation}
\mathcal{I} = \{y_1, y_2, \ldots, y_c\}, \qquad \mbox{where} \qquad y_1, y_2,
\ldots, y_c \in \mathbb{R}^m
\end{equation}
%
A sample classification problem is shown in
figure \ref{fig:classification problem}.

Thus, the classification problem involves splitting the input space
into a set of disjoint regions $\{ \mathcal{R}_1 \ldots \mathcal{R}_c
\}$ which correspond to the category values.  A useful representation
of these regions is gained by constructing the ``decision boundaries''
as illustrated in figure \ref{fig:classification problem}.

\begin{linefigure}
\begin{center}
\includegraphics{figures/classification_problem.epsg}
\end{center}
\caption{A classification problem}
\label{fig:classification problem}
The $\times$ symbol represents $y=-1$ and the $\circ$ symbol $y=1$.
The dotted line is the ``decision boundary'', which is the boundary of
adjacent regions of different symbols.  Note that in the data shown,
there are many possible decision boundaries which classify the data
correctly.
\end{linefigure}

For the remainder of this thesis, only classification problems will be
considered.  Unless otherwise specified, these will be \emph{binary}
classification problems (two categories) where $\mathcal{O} = \{-1,
1\}$.

\subsection{Sign formulation}
To aid analysis, equation \ref{eqn:learning machine formulation}
can be recast into the form
%
\begin{equation}
y = \sign \left( f(\mathbf{x}) \right)
\label{eqn:marginal classification formulation}
\end{equation}
%
where $f : \mathbb{R}^n \mapsto \mathbb{R}$ and
%
\begin{equation}
\sign (x) = \left\{ \begin{array}{rl}
1	& \qquad \mbox{if $x \geq 0$} \\
-1	& \qquad \mbox{otherwise}
\end{array} \right.
\end{equation}
%
It is from this definition that the concept of a ``margin'' is
developed.  This is explored further in section \ref{sec:the margin}.


\subsection{Empirical risk minimisation}
\label{sec:erm}

Now that we have defined what we mean by a ``learning algorithm'' (and
in particular, what we mean by a ``classifier'', we need a way of
comparing two classifiers to determine which is ``better''.


\subsubsection{Loss function}

In order to make such a comparison, we need to introduce the concept
of a ``loss function''.  The input to a loss function is a triple
$(\bfx, \bfyh, \bfy)$ where \bfx\ is a sample, \bfyh\ is the
output of the classifier (learning machine), and \bfy\ is the correct output (that generated by the supervisor).  See figure \ref{fig:supervised
learning}.  The output of the loss function is a single number $q$
which describes how ``bad'' this estimate is.  In symbols, the loss
function can be described as
%
\begin{equation}
q = Q(\bfx, \bfyh, \bfy) \qquad \mbox{where} \qquad q \in \mathbb{R};
\bfx \in \calI; \mbox{ and }\bfyh, \bfy \in \calO
\end{equation}
%
where all symbols have their previously defined meanings.

In this thesis, only one loss function will be considered.  This is
known as the ``misclassification loss function'', and is defined as
%
\begin{equation}
Q(\bfx, \bfyh, \bfy) = Q(\bfyh, \bfy) = \left\{
\begin{array}{ll}
	0	&	\qquad \mbox{if $\bfyh = \bfy$} \\
	1	&	\qquad \mbox{if $\bfyh \neq \bfy$}
\end{array}
\right.
\label{eqn:misclassification loss function}
\end{equation}
%
which does \emph{not} depend upon the position in the input space \bfx.

This loss function is very easy to understand: given a series of input
samples, the sum of the loss functions will be equal to the number of
samples that were misclassified.  If the performance of two
classifiers are compared using this loss function, then the one with
the lowest risk will be the one that makes the least ``mistakes''.


\subsubsection{True risk}
Given our loss function $Q(\cdot)$, we are now in the position to be
able to compare the performance of a number of classifiers over the
entire input space.  We define the \emph{true risk} of a classifier
$\hat{h}(\bfx)$ given a supervisor $h(\bfx)$ as
%
\begin{equation}
R(\hat{h}) = \int_{\calI} Q(\bfx, \bfyh, \bfy) \: d\bfx
\label{eqn:true risk}
\end{equation}
%
Using our loss function \ref{eqn:misclassification loss function}, the
value of $R$ is easily seem to be the proportion of samples that are
misclassified.

The goal of machine learning theory is generate learning
machines that minimise the true risk \ref{eqn:true risk}.
Unfortunately, this is impossible for at least two reasons:
%
\begin{enumerate}
\item	Generally, \ref{eqn:true risk} cannot be solved analytically,
	or in finite time, making the minimisation of it impossible.
%
\item	In practice, there are only a finite number of observations
	$(\bfx_1, \bfy_1), \ldots, (\bfx_l, \bfy_l)$ available.  Thus
	it is impossible to evaluate \ref{eqn:true risk}.
\end{enumerate}
%
For this reason, we are reliant on the concept of \emph{empirical
risk}.

\subsubsection{Empirical risk}
Empirical risk is a weaker version of true risk, that is actually able
to be calculated in practice.  Given a set of $l$ observations
$\{(\bfx_1, \bfy_1), \ldots, (\bfx_l, \bfy_l)\}$ and setting $\bfyh_i
= \hat{h}(\bfy_i)$ as the output of our learning machine $h(\cdot)$,
we can define our empirical risk as 
%
\begin{equation}
R_{\emp} = \frac{1}{l} \sum_{i=1}^{l} Q(\bfx_i, \bfy_i, \bfyh_i)
\end{equation}
%
The following observations can be made on the empirical risk:
%
\begin{itemize}
\item 	As $l \rightarrow \infty$, and assuming that $\bfx$ is evenly
	distributed over \calI\, (and that $Q$ is reasonably well
	behaved), we would expect that $R_{\emp}(\cdot) \rightarrow
	R(\cdot)$.

\item	$R_{\emp}$ is the ``best that we can do'', in that it uses all
	available observations.  However, there is further scope for
	\emph{a priori} knowledge of the system to be used via the
	selection of an appropriate model.

\item	Section \ref{sec:slt} explores the rate of convergence of
	empirical risk $R_{\emp}$ to true risk $R$.
\end{itemize}

\subsubsection{Empirical risk minimisation}

Until this point, we have been concerned with a \emph{single} learning
machine $\hat{h}(\bfx)$.  We are really concerned with a set of such
learning machines $\hat{h} \in \calH$, from which we choose the
optimal learning machine $\hat{h}^{\ast}$.  This is
formalised in the ``empirical risk minimisation'' inductive principle:

Assume that we have a set of learning machines
%
\begin{equation}
\hat{h}_i \in \calH \qquad \mbox{where} \qquad h_i : \calI \mapsto
\calO
\end{equation}
%
and a loss function $Q$.

Then the principle of \emph{empirical risk minimisation} selects the
optimal learning machine $\hat{h}^{\ast} \in \calH$ in the sense that
the empirical risk is minimised:
%
\begin{equation}
\hat{h}^{\ast} = \arg \min_{\hat{h} \in \calH} R_{\emp}(\hat{h})
\label{eqn:erm}
\end{equation}

In order to comment further on this principle, we need to develop some
ideas from statistical learning theory.  This is covered in the next
section.






\section{Statistical learning theory}
\label{sec:slt}

This section develops important results from statistical learning
theory (SLT).  SLT provides a mathematically rigorous
conceptual framework in which to evaluate the many empirical results.
This theory is covered in much greater detail by Vapnik
\cite{Vapnik98} and Cherkassky \& Mulier \cite{Cherkassky98}.  Results
are also taken from Bartlett et al. \cite{Bartlett98a}.


\subsection{Convergence of ERM}

The goal of this section is to determine under which conditions the
empirical risk minimisation inductive pronciple defined in section
\ref{sec:erm} will converge to the true risk \ref{eqn:true risk}.
This theorem, which is known as the ``key theorem of learning theory''
appears in \cite{Cherkassky98}.

\subsubsection{Key theorem of learning theory}

For bounded loss functions, the ERM principle is consistent if and
only if the empirical risk converges uniformly to the true risk in the
following sense:
%
\begin{equation}
\lim_{l \rightarrow \infty} \Pr \left( \sup_{\hat{h} \in \calH} 
| R(\hat{h}) - R_{\emp}(\hat{h}) | > \epsilon \right) = 0,
\qquad \forall \epsilon > 0
\end{equation}
%
The most important result of this theory in a practical sense is the
assertion that consistency is governed by the \emph{worst case}
scenario (selected by the $\sup$ operator).

This theorem is not useful for developing constructive procedures.  A
more useful theorem is developed by considering the \emph{entropy} of
a set of functions.

Consider a loss function $Q(\bfx, \bfyh, \bfy)$, our set of
classifiers $\calH$, and a given set of samples $Z_n = \{ (\bfx_1,
\bfy_1), \ldots, (\bfx_n, \bfy_n) \}$ where $n$ now denotes the number
of samples in $Z_n$.  We would expect that different classifiers
$\hat{h} \in \calH$ would split these data points into different
partitions (see figure \ref{fig:dichotomy}).

\begin{linefigure}
\begin{center}
\includegraphics{figures/dichotomy.epsg}
\end{center}
\caption{Two classifiers $\hat{h}_1$ and $\hat{h}_2$ partitioning a
dataset into two dichotomies.}
\label{fig:dichotomy}
\end{linefigure}

We denote the number of such dicotomies that can be generated by a
particular set of classifiers $\calH$ on a sample $Z_n$ by $N(Z_n)$.
This is known as the \emph{diversity} of the set of functions $\calH$.

\subsubsection{Random entropy}

The random entropy can then be defined as
%
\begin{equation}
H(Z_n) = \log N(Z_n)
\end{equation}

\subsubsection{VC entropy}

Averaging this quantity over all possible samples of length $n$
generated from the sample distribution leads to the \emph{VC entropy}
of the set of classifiers:
%
\begin{equation}
H(n) = E(\log N(Z_n))
\end{equation}

\subsubsection{Growth function}

The final quantity that we need to define is that of the growth
function.  This is a distribution-independent quantity (independent of
the distribution that the training samples are drawn from), and is
defined as
%
\begin{equation}
G(n) = \log \max_{Z_n} N(Z_n)
\end{equation}
%
where the samples $Z_n$ are taken over \emph{all} possible samples of
size $n$ regardless of the distribution.  The growth function is the
maximum number of dichotomies that can be deduced on a sample of size
$n$ using the classifiers $\hat{h}$ in a set $\calH$.

Due to the $\max$ function, the growth function will always be at
least as large as the VC entropy function.  The growth function will
also be bounded above by $2^n$, the maximum number of dichotomies that
can be generated on $n$ samples.  Thus, the following inequality holds:
%
\begin{equation}
H(n) \leq G(n) \leq n \log 2
\end{equation}

\subsubsection{ERM consistency}

A useful, distribution-independent, necessary \emph{and} sufficient
condition for the consistent convergence of ERM is
given by
%
\begin{equation}
\min_{n \rightarrow \infty} \frac{G(n)}{n} = 0
\label{eqn:growth function ERM consistency}
\end{equation}
%
This condition requires that at some point, the growth function $G(n)$
must flatten out (by adding another sample, we don't double the number
of dichotomies that can be generated).

Equation (\ref{eqn:growth function ERM consistency}) also guarantees
that the rate of convergence is \emph{fast}, in that for any $n >
n_0$ the following exponential bound holds true: 
\begin{equation}
\Pr \left( R(\hat{h}) - R(\hat{h}^{\ast}) < \epsilon \right) =
e^{-cn\epsilon^2}
\end{equation}
 


\subsection{Vapnik-Chervonenkis Dimension}
\cite{Cherkassky98} section 4.2 and \cite{Bartlett98a}

\subsubsection{VC upper bound}

Let $H$ be a class of functions mapping from a set $X$ to $\{-1, 1\}$
and having VC-dimension $h$.  For any probability distribution on $X
\times \{-1,1\}$, with probability $1-\delta$ over $l$ random examples
$\mathbf{x}$, any hypothesis $f$ in $H$ has generalisation error no
more than
\begin{equation}
2\mathrm{Er}_{\mathbf{x}}(f) + \frac{1}{l} \left[ 4 \ln 
\left( \frac{4}{\delta} \right) + 4 h \ln \left( \frac{2 e l}{h}
\right) \right]
\end{equation}
provided $h \leq l$.

\subsubsection{VC lower bound}

Let $H$ be a hypothesis space with finite VC dimension $h \geq 1$.
Then for any learning algorithm there exist distributions such that
with probability at least $\delta$ over $l$ random examples, the error
of $f$ is at least
\begin{equation}
\max \left[ \frac{h-1}{32l}, \frac{1}{l} \ln \left( \frac{1}{\delta}
\right) \right]
\end{equation}

\noindent Notes:

\begin{itemize}

\item	The upper bound holds for all distributions, whereas the lower
	bound only holds for points that can be ``shattered'' by the
	hypothesis space.

\item	The upper bounds appear to be very pessimistic in practice.
	Possibly real-world data does not behave like the worst-case
	distributions used to prove the lower bounds.

\end{itemize}


\subsubsection{Structural risk minimisation}

Suppose that $X$ is the ball of radius $R$ in $\mathbb{R}^n$,
$X = \left\{ \mathbf{x} \in \mathbb{R}^n : \| \mathbf{x} \| \leq R
\right\} $, and that $X_0 \subset X$.  Consider the set
\begin{equation}
\mathcal{F} = \left\{ \mathbf{x} \mapsto \mathbf{w} \cdot \mathbf{x} :
\|\mathbf{w}\| \leq 1, \mathbf{x} \in X \right\}
\end{equation}
and let $\mathrm{sign}(\mathcal{F}) = \left\{ \mathbf{x} \mapsto
\mathrm{sign}(f(\mathbf{x})) : f \in \mathcal{F} \right\}$ be the set of
classifiers defined by thresholding functions in $\mathcal{F}$ that
satisfy $|f(\mathbf{x})| \geq \gamma$ for all $\mathbf{x}$ in $X_0$.
Then the restriction of $\mathrm{sign}(\mathcal{F}_0)$ to the points
in $X_0$ has VC dimension no more than
\begin{equation}
\min \{R^2/\gamma^2, n\} + 1
\end{equation}


\subsection{Bounds on Generalisation Error}
\cite{Cherkassky98} section 4.3

Let us assume that we have been given the following information:

\begin{itemize}

\item	A sample $\{\mathbf{x}_i, y_i\} = \{\mathbf{z}_i\}$ of size
	$i=1 \ldots n$;

\item	A classification function $y = f(\mathbf{x}, \omega)$ where
	$\omega$ is a parameter which distinguishes a particular
	instance of the classifier from the whole family;

\end{itemize}

Now consider the loss function $Q(\mathbf{z}, w^* | n)$ minimising the
empirical risk.  Let $R_{\mathrm{emp}}(\omega^* | n)$ denote this
minimum empirical risk, and let $R(\omega^* | n)$ denote the true risk
corresponding to this loss function.  Then we can say the following:

\begin{itemize}

\item	In a binary classification problem, the generalisation ability
	of a learning machine that implements ERM holds with a
	probability of \emph{at least} $1-\eta$ simultaneously for
	all functions $Q(\mathbf{z}, \omega)$ including the function
	$Q(\mathbf{z}, \omega^*)$ then minimises the empirical risk:

	\begin{equation}
	R(\omega) \leq R_{\emp}(\omega) + \frac{\epsilon}{2} \left(
	1 + \sqrt{1 + \frac{4 R_{\emp}(\omega)}{\epsilon}} \right)
	\end{equation}

	where

	\begin{equation}
	formula
	\end{equation}

\end{itemize}


\subsection{Structural Risk Minimisation}
\cite{Cherkassky98} section 4.4







\section{The margin}
\label{sec:the margin}

The margin of a classification using the notation of \ref{eqn:marginal
classification formulation} is given by
%
\begin{equation}
m_{\mathbf{x}} = \left\{ \begin{array}{rl}
yf(\mathbf{x})	& \qquad \mbox{if $y = \hat{y}$} \\
-yf(\mathbf{x})	& \qquad \mbox{if $y \neq \hat{y}$}
\end{array} \right.
\label{eqn:margin definition}
\end{equation}

Geometrically, this may often be considered as the distance between a
point and the decision boundary at that particular point\footnote{Need
to point out under what circumstances this is true and untrue, eg for
``well behaved'' functions which include all? that we are talking
about here}.  This is illustrated in figure \ref{fig:margin
illustration}.

\begin{linefigure}
\begin{center}
\includegraphics{figures/margin_illustration.epsg}
\end{center}
\caption{The margin of a sample}
\emph{The dotted line denotes the decision boundary and the $\times$ the
point for which we wish to define a margin.}
\label{fig:margin illustration}
\end{linefigure}

The margin of an entire dataset $\mathbf{X} = \{ \mathbf{x}_1, \ldots,
\mathbf{x}_l \}$ is defined as the minimum margin over all
samples:\footnote{Is this true?  Is it useful?}
%
\begin{equation}
m_{\mathbf{X}} = min \left( m_{\mathbf{x}_1}, \ldots, m_{\mathbf{x}_l}
\right)
\label{eqn:dataset margin}
\end{equation}
%

There are two definitions of a margin commonly used in a
classification problem:

\subsubsection{Minimum distance margin}
Consider a classifier that generates margins using a ``voting''
mechanism.  The $c$ votes $v_1, \ldots, v_c$ are, and the classifier
selects the category with highest vote $v^{\ast}$ as the result of the
classification.   Then the minimum distance margin is the difference
between the votes of the two kinds of margins:
%
\begin{equation}
mdm = v^{\ast} - max_{v_i \in \mathbf{v} - v^{\ast}} v_i \qquad
\mbox{where} \qquad v^{\ast} = max_{v_i \in \mathbf{v}} v_i
\end{equation}
%
or in other words, the difference between the vote $v^{\ast}$ that the
highest category got, and  the vote that the next-highest category
got.

Unfortunately, the discontinuity introduced by the $\max$ function
makes it difficult to solve
problems analytically using the minimum distance margin.  The other
margin provides a similar measure, but is more suited to analysis.

\subsubsection{The other margin}
Consider the same margin problem.  Then the other margin is generated
by the formula
%
\begin{equation}
otm = function
\end{equation}
%
etc. 


\subsection{Fat-shattering dimension}
These are copied verbatim from Bartlett et al. \cite{Bartlett98a}.

Let \calF\ be a set of real values functions.  We say that a set of
points $X$ is \emph{$\gamma$-shattered by \calF} if there are real
numbers $r_x$ indexed by $x \in X$ such that for all binary vectors
$b$ indexed by $X$, there is a function $f_b \in \calF$ satisfying
\begin{equation}
f_b(x)  \left\{
	\begin{array}{ll}
		\geq r_x + \gamma & \mbox{if $b_x = 1$} \\
		\leq r_x - \gamma & \mbox{otherwise}
	\end{array}
\right.
\end{equation}
The \emph{fat shattering dimension} $\fat_{\calF}$ of the set \calF\ is
a function from the positive real numbers to the integers which maps a
value $\gamma$ to the size of the largest $\gamma$-shattered set (if
this is finite), or infinity otherwise.


\subsection{Margin bounds}
The following theorem gives generalisation error bounds for large
margin classifiers.

Consider a class \calF\ of real-valued functions.  With probability at
least $1 - \delta$ over $l$ independently generated examples
$\mathbf{z}$, if a classifier $\sign (f) \in \sign (\calF)$ has margin
at least $\gamma$ on $\mathbf{z}$, then the error of $\sign (f)$ is no
more than
\begin{equation}
\frac{2}{l} \left[ h \log_2 \left( \frac{8 e l}{h} \right) \log_2(32l)
+ \log_2 \left( \frac{8l}{\delta} \right) \right]
\end{equation}
where $h = \fat_{\calF}(\gamma / 16)$.

Furthermore, with probability at least $1 - \delta$, every classifier
$\sign (f) \in \sign (\calF)$ has error no more than
\begin{equation}
b/l + \sqrt{\frac{2}{l} \left[ h \ln(34el/h) \log_2(578l) +
\ln(4/\delta) \right] }
\end{equation}
where $b$ is the number of labelled training examples with a margin
less than $\gamma$.

See the explanation of this in \cite{Bartlett98a} pages 5-6.



\section{Covering numbers and other stuff}
At the moment, this whole section is copied pretty much verbatim from
\cite{Williamson99}.

The p-convex hull (strictly speaking, the p-absolutely convex hull) of
a set $S$ of functions (where $p>0$) is defined as
%
\begin{equation}
\cop (S) =
 \bigcup _{n \in \mathbb{N}}
\left\{
 \sum_{i=1}^{n}
 \alpha _i
f_i : f_1, \ldots, f_n \in F,
 \alpha _1, \ldots, \alpha _n \in \mathbb{R},
 \sum_{i=1}^{n} | a_i |^p \leq 1
\right\}
\end{equation}

Thus the $p$-convex hull is a subset of the set of all possible linear
combinations of functions of a class $F$, where the members of the
subset are those where $\sum_{i=1}^n |\alpha_i| \leq 1$.  The
following remarks concern $p$-convex hulls:
%
\begin{itemize}
\item	If $0 < p_1 \leq p_2 \leq \cdots \leq p_z \leq 1$, then
	$\cop 1 \subseteq \cop 2 \subseteq \cdots \subseteq \cop z$
	\marginpar{\em Make sure this is true!}.  That is to say, the
	higher that $p$ is, the richer the set of functions in
	$\cop{F}$.
\item	In general, it is very hard to visualise the \emph{convex} hull
	of a set of functions, let alone the $p$-convex hull.  The
	convex hull is always richer than function class.
\item	As an example (from Williamson et al. \cite{Williamson99}), the convex
	hull of the Heaviside functions on $[0, 1]$ is the set of
	functions of ``bounded variation''.\marginpar{Need a diagram}
\end{itemize}










