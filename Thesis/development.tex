% development.tex
% Jeremy Barnes, 1999
% $Id$

\chapter{Development of algorithms}

\section{Scale invariance}

The boosting algorithm is sensitive only to the relative magnitude of
the classifier weights $b_i$, not to the absolute magnitude.

To see this, consider the linear combination of weights
%
\begin{equation}
y = \sign(F(\bfx)) = \sign(b_1 f_1(\bfx) + b_2 f_2(\bfx) + \cdots +
b_T f_T(\bfx)
\end{equation}
%
Now as
%
\begin{equation}
\sign(\alpha x) = \sign(x) \qquad \qquad \mbox{for $\alpha \neq 0$}
\end{equation}
%
a linear combination that is scaled by an $\alpha \neq 0$
%
\begin{equation}
y = \sign(F(\bfx)) = \sign(\alpha b_1 f_1(\bfx) + \alpha b_2 f_2(\bfx)
+ \cdots + \alpha b_T f_T(\bfx)
\end{equation}
%
is exactly the same as before.

For this reason, normalising of the weight vector is unnecessary.  It
also means that normalising to \emph{any} norm is equivalent to any
other norm; in particular, we will get equivalent behaviour by
normalising in a $p$-norm as we will in a $1$-norm.

\section{Refinement of gradient descent}

Once we have chosen $f_{t+1}$, we need to choose $w_{t+1}$.  This is
chosen to minimise the cost functional along the line given by the
direction $f_{t+1}$.  We can write the value of this cost functional
as
%
\begin{equation}
C = \sum_{i=1}^{m} c \left( y_i F_t(\bfx_i) + y_i w_{t+1}
f_{t+1}(\bfx_i) \right)
\end{equation}
Substituting in $c(f(x)) = e^{-yf(x)}$, we obtain
%
\begin{equation}
C = \sum_{i=1}^{m} \exp \left\{ y_i F_t(\bfx_i) + y_i w_{t+1}
f_{t+1}(\bfx_i) \right\}
\end{equation}
%
Now we need to differentiate this expression with respect to
$w_{t+1}$.   This task is simplified considerably by noting that only
the second half of the exponential term is variable with $w_{t+1}$.
The result is that
%
\begin{eqnarray}{ll}
\frac{\partial C}{\partial w_{t+1}} &
= & \sum_{i=1}^{m} \frac{\partial}{\partial w_{t+1}}
\exp \left\{ y_i F_t(\bfx_i) + y_i w_{t+1}
f_{t+1}(\bfx_i) \right\} \nonumber \\
& = & \sum_{i=1}^{m} y_i \exp \left\{ y_i F_t(\bfx_i) + y_i w_{t+1}
f_{t+1}(\bfx_i) \right\} 
\end{eqnarray}



