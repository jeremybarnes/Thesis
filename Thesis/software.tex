% software.tex
% Jeremy Barnes, 22/10/1999
% $Id$

% This file briefly describes the computer software that I developed for
% this project, and provides some examples of its use.

\chapter{Software}
\label{appendix:software}

\section{Introduction}

\section{Installation}

\section{Classes and functions}

\subsection{Dataset}
The {\tt dataset} class encapsulates a dataset with several input
attributes and one output 

\begin{tabular}{ll}
\hline
dataset 	& Create a dataset object \\
\hline
numcategories 	& Return number of categories (binary=2)\\
numsamples 	& Return number of samples \\
dimensions	& Return number of dimensions (attributes) \\
\hline
data		& Return samples and labels \\
x\_values	& Return samples \\
y\_values	& Return labels \\
\hline
datagen		& Generate data from a distribution \\
load		& Load a dataset \\
addsamples	& Add data to the dataset \\
\hline
partition	& Split dataset into two parts \\
randomize	& Randomly permute the order samples \\
\hline
dataplot	& Generate a plot of the dataset \\
\hline
\end{tabular}

\subsection{Classifier}

The classifier type is the ancestor of all of the learning machines.
The following functions apply to both the classifier type and all of
its descendent types.

\begin{tabular}{ll}
\hline
classifier		& Create a classifier object \\
\hline	
dimensions		& Return dimensions of input space \\		
numcategories		& Return number of categories \\
\hline
train			& Train a classifier \\
classify		& Classify a set of samples \\
empirical\_risk		& Return the empirical risk over a dataset \\
margins			& Return the margins of a set of samples \\
\hline
trained\_samples		& Return the samples trained over \\
training\_error		& Return the empirical risk over training dataset \\
\hline
marginplot		& Plot margins of a dataset \\
plot\_classification	& Plot how samples get classified \\
plot\_decision\_boundary	& Approximate and plot decision boundary from margins \\
\hline
\end{tabular}{ll}


\subsection{CART}

The CART classifier is a full decision-tree based algorithm (a
generalisation of Decision Stumps).  In addition to the functions
inherited from classifier, it includes the following functions.

\begin{tabular}{ll}
\hline
cart			& Create a CART classifier \\
\hline
classify		& Classify some data \\
classification\_error	& Return error of previous classify operation \\
\hline
train			& Train CART classifier \\
training\_error		& Return empirical risk over training dataset \\
\hline
plotboundary		& Equivalent to plot\_decision\_boundary \\
plottree		& Divide plane into coloured areas \\
printtree		& Print textual representation of tree \\
treesize		& Return number of leaf nodes in tree \\
\hline
\end{tabular}

\subsection{Decision stump}

\begin{tabular}{ll}
\hline
decision\_stump		& Create a decision stump classifier \\
\hline
margins			& Return margins \\
marginplot		& Plot margins \\
\hline
plottree		& Graphical representation of decision tree \\
printtree		& Textual representation of decision tree \\
treesize		& Return number of leaf nodes (always 2) \\
\hline
\end{tabular}

\subsection{Neural network}

\begin{tabular}{ll}
\hline
neural\_net		& \\
\hline
classify		& \\
\hline
train			& \\
trainagain		& \\
trainfirst		& \\
training\_data		& \\
training\_error		& \\
training\_labels	& \\
training\_samples	& \\
\hline
test			& \\
\hline
get			& \\
set			& \\
\hline
hidden\_units		& \\
learning\_rate		& \\
momentum		& \\
\hline
hidden\_weights		& \\
output\_weights		& \\
iterations		& \\
\hline
\end{tabular}

\subsection{Boost}

abort
aborted
add\_iteration
as\_boost
as\_classifier
boost
classifier\_weights
classify
iterations
margins
sample\_weights
test
train
trainagain
trained\_samples
trainfirst
training\_error
update\_margins
w
weaklearner
weight\_density\_plot
wl\_instance
x
y

\subsection{Strict $p$-boosting algorithm}

eval\_cf
norm
normboost
trainagain
update\_margins

\subsection{Sloppy $p$-boosting algorithm}

eval\_cf
normboost2
trainagain
update\_margins

\subsection{Na\"{\i}ve $p$-boosting algorithm}

as\_boost
b\_plot
boost
cvs
disp
disp\_info
get\_p
p\_boost
trainagain


\subsection{Visualisation functions}

density\_plot
load\_pgm
plot\_domain
plot\_margin\_distribution
plot\_region
twoclass\_colormap
twoclass\_density\_to\_color
uniform\_color\_cube


\section{Testing and experiments}

\begin{tabular}{ll}
\hline
maketest		& \\
\hline
runtest			& \\
\hline
summarise		& \\
summarise\_boost	& \\
meta\_summarise		& \\
\hline
get\_test\_info		& \\
get\_test\_results	& \\
\hline
display\_meta\_summary	& \\
display\_results	& \\
plot\_results		& \\
\hline
\end{tabular}