% notation.tex
% Jeremy Barnes, 1999
% $Id$

\chapter{Acronyms and Notation}

\section*{Acronyms}

\begin{tabular}{l l l}

\bf{Acronym} & \bf{Meaning} & \bf{First introduced} \\ \hline \hline

SLT	& Statistical Learning Theory 	& \ref{acr:slt} \\
ERM	& Empirical Risk Minimisation 	& \ref{acr:erm} \\
SRM	& Structural Risk Mininmisation & \ref{acr:srm} \\
VC dimension & Vapnik-Chervonenkis dimension & \ref{acr:vcdim} \\
\hline

\end{tabular}

\section*{Notation}
\newcommand{\longexp}[1]{\parbox[t]{3in}{#1}}
\begin{tabular}{l l c}
\bf{Symbols}		& \bf{Meaning}		& \bf{Section} \\
\hline \hline
$\bfx \in \calI$	& Sample
			& fig \ref{fig:supervised learning} \\

$y \in \calO$		& Label
			& " \\

$X = ((\bfx_1, y_1), \ldots))$
			& Labeled samples, length $m$ (training dataset)
			& \ref{sec:formulation} \\

$\calI, \calO$		& Domain and range (usually $\{\pm 1\}$) of problem
			& \ref{sec:domain and range} \\
\vspace{5mm}
$h(\cdot) \in \calH : \calI \mapsto \calO$
			& A hypothesis (classifier)
			& \ref{sec:formulation} \\

$\calH$			& A set of possible hypotheses $h$
			& \ref{representation learning machines} \\

$f(\cdot) \in \calF : \calI \mapsto \bbR$
			& \longexp{A non-thresholded hypothesis; usually 
			  $h(\cdot) = \sign(f(\cdot))$}
			& \ref{sec:margin formulation} \\

$\calF$			& A set of possible hypotheses $f$ 
			  ($\calH = \sign(\calF)$)
			& " \\

$\bbW : \calI^m \mapsto (\calI \mapsto \calO)$
 			& Learning machine ($\bbW(X) = h$)
			& " \\
\vspace{5mm}
$q = Q(\bfx, y, \hat{y})$
			& Loss function
			& \ref{sec:loss function} \\

$R(h)$			& True risk
			& \ref{sec:true risk} \\

$R_{\emp}(h)$		& Empirical risk
			& \ref{sec:empirical risk} \\

$R_{\emp}^w(h)$		& Weighted empirical risk (training error)
			& \ref{sec:weighted empirical risk} \\

$R_{\emp}^{\gamma}(h)$	& Margin risk
			& \ref{sec:margin risk} \\

\vspace{5mm}
$\VCdim(\cdot)$		& VC dimension
			& \ref{sec:vcdim} \\

$\covert{\calX}{\epsilon}{d(\cdot, \cdot)}$ &
			\longexp{Covering number at scale $\epsilon$ of $\calX$
			using norm $d$ (usually $d_{\infty}$; assumed if not
			specified)}
			& \ref{sec:covering numbers} \\

$\covert{\calX}{\epsilon}{k}$ &
			\longexp{
			Uniform covering number of $\calX$ at scale $\epsilon$
			over length $k$}
			& " \\

$\co_{p}(\calX)$	& $p$-convex hull of set $\calX$
			& \ref{sec:p-convex} \\
\hline
\end{tabular}
\par\par\noindent
A hat ($\hat{\ \ }$) means ``estimate''.  An asterisk ($\ \ ^{\ast}$) means
``optimal'' in some sense.
