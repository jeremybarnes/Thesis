% datasets.tex
% Jeremy Barnes, 16/10/1999
% $Id$

\chapter{Description of datasets used}
\label{appendix:datasets}

This appendix contains a brief description of the datasets used in the
experiments.  Details of the problem domain, descriptions of the
attributes, and the results of previous studies using these datasets
are given.

\section{Ring dataset}

The \ds{ring} dataset is generated from a two dimensional mathematical
distribution on the unit square.  The sample probability density is
constant within the unit square:
%
\begin{equation}
p((x_1, x_2)) = \left\{ \begin{array}{rr}
1	& \qquad \mbox{if $(x_1, x_2) \in [0,1]^2$} \\
0	& \qquad \mbox{otherwise}
\end{array} \right.
\end{equation}
%
and the label value is $+1$ within a circle of radius $\sqrt{1/8}$
centred at $(\frac{1}{2},\frac{1}{2})$, or $-1$ otherwise.  The
decision boundary of this dataset is plotted in figure \ref{fig:ring
distribution}.

\begin{linefigure}
\begin{center}
\includegraphics{figures/ring_distribution}
\end{center}
\begin{capt}{The \ds{ring} distribution}{fig:ring distribution}
The markers are sample data points ($\times=+1$, $\bullet=-1$).  The
decision boundary is plotted in the dashed line.
\end{capt}
\end{linefigure}

The \ds{ring10}, \ds{ring20} and \ds{ring30} datasets have noise added
artificially: a proportion of samples $n$ is selected \emph{without
replacement} from the generated data, and each of these samples has
its label ``flipped''.

No previous experiments have been performed using this data.  However,
as no noise is added to the \emph{test} datasets used, it is possible
for the test error to reach 0 if the algorithm generalises perfectly.

\section{Acacia dataset}

The \ds{acacia} dataset is an example of ecological/geographical
data.  The attributes of the dataset are listed in table
\ref{tbl:acacia attributes}; there were 204 samples.  The intended
application of the dataset was to generate a learning machine that
could detect the presence or otherwise of certain species from site
and satellite data.

The \ds{acacia} dataset considered in this thesis used a subset of the
data; in particular columns 19 through 21 were ignored; the two-class
classification problem being detecting the presence of {\it Acacia
mabellae} (column 19) from the site attributes (columns 1 through
18).  Column 3 (the site number) was removed from the dataset, leaving
a total of 17 attributes.  (The {\it Acacia mabellae} species was
chosen because it had a roughly even number of positive and negative
examples (119 and 85).

Previous work with this dataset \cite{Payne97, Williamson97a} used a
Support Vector Machine \cite{Vapnik98}.  The average test error over
100 trials for a partition of 144 training samples and 60 test samples
was 32.6\%.

\begin{table}
\begin{center}
\begin{tabular}{l l}
\hline
\bf{No.} & \bf{Description} \\
\hline\hline
1&	easting \\
2&	northing \\
3&	site number \\
4&	elevation  \\
5&	drainage area \\
6&	slope (percent)\\
7&	flow path length\\
8&	landsat band 1 \\
9&	landsat band 2\\
10&	landsat band 3\\
11&	landsat band 4 \\
12&	landsat band 5\\
13&	landsat band 6 \\
14&	landsat band 7 \\
15&	nutrient supply index\\
16&	geology\\
17&	net radiation\\
\hline
18&	{\it Acacia mabellae} present\\
19&	{\it Breynia oblongifolia} present\\
20&	{\it Eucalyptus maculata} present\\
21&	{\it Eucalyptus tereticornis} present\\
\hline
\end{tabular}
\end{center}
\caption{Attributes of the \ds{acacia} dataset}
\label{tbl:acacia attributes}
\end{table}

\section{Sonar dataset}

This dataset (from the UCI repository \cite{UCI}) contains data
obtained by bouncing sonar signals off two objects: a metal cylinder
and rock, under similar conditions.  The sonar signal is preprocessed
into 60 values in the range $[0,1]$ which each measure the energy in a
particular frequency band.  The experiment was repeated with the
cylinders rotated at different angles.

The two class classification problem is to detect whether the object
is a rock or metal cylinder.  There are 208 examples in the dataset,
comprising 97 cylinders and 111 rocks.

Previous experiments using this dataset \cite{Gorman88} using a Neural
Network over 13 trials of 192 training samples and 16 test samples
resulted in a test error of 15.3\%.


\section{Wisconsin prognostic breast cancer dataset}

The \ds{wpbc} dataset is also from the UCI repository \cite{UCI}, and
comes from the medical domain.  It contains prognostic information for
breast cancer patients of a Wisconsin doctor%
\footnote{Of course, the patient details are not identifiable through
the dataset.}.

A total of 32 attributes are recorded, which contain various
summarised information on the status of the patient, their previous
history, and cell nuclei in breast tissue samples.  The machine
learning problem is to predict whether the breast cancer will recur.
The dataset contains a total of 198 samples, of which 151 of the
samples are (non-recurring) and 47 positive (recurring).  Of these 198
samples, 4 have a missing attribute; these samples are ignored to give
a total of 194 samples.
