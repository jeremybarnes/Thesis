% stumps.tex
% Jeremy Barnes, 22/9/1999
% $Id$

% commands.tex
% Jeremy Barnes, 1999
% $Id$

\providecommand{\emp}{\mathrm{emp}}
\providecommand{\calF}{\ensuremath{\mathcal{F}}}
\providecommand{\fat}{\ensuremath{\mathrm{fat}}}
\providecommand{\sign}{\ensuremath{\mathrm{sign}}}
\providecommand{\cop}{\ensuremath{\mathrm{co}_p}}
\providecommand{\co}{\ensuremath{\mathrm{co}}}
\providecommand{\bfx}{\ensuremath{\mathbf{x}}}
\providecommand{\bfy}{\ensuremath{\mathbf{y}}}
\providecommand{\bfyh}{\ensuremath{\hat{\mathbf{y}}}}
\providecommand{\calO}{\ensuremath{\mathcal{O}}}
\providecommand{\calI}{\ensuremath{\mathcal{I}}}
\providecommand{\calH}{\ensuremath{\mathcal{H}}}
\providecommand{\calX}{\ensuremath{\mathcal{X}}}
\providecommand{\ip}[2]{\ensuremath{\langle {#1} , {#2} \rangle}}
\providecommand{\lin}{\mathrm{lin}}
\providecommand{\calS}{\ensuremath{\mathcal{S}}}
\providecommand{\VCdim}{\mathrm{VCdim}}
\providecommand{\Fat}[1]{\mathrm{Fat}_{#1}}
\providecommand{\cover}[2]{\mathcal{N}({#1}, {#2})}
\providecommand{\covert}[3]{\mathcal{N}({#1}, {#2}, {#3})}
\providecommand{\MATLAB}{{\tt MATLAB}}
\providecommand{\C}{{\tt C}}
\providecommand{\argmin}{\mathrm{argmin}}

% Theorem-like constructs
\newtheorem{theorem}{Theorem}
\newtheorem{definition}{Definition}

\providecommand{\proof}{\par \par \noindent {\bf Proof:\ }}

\providecommand{\figlinewidth}{1pt}

\newenvironment{linefigure}%
		{\begin{figure} \rule{\textwidth}{\figlinewidth}}%
		{\rule{\textwidth}{\figlinewidth}\end{figure}}


\chapter{Decision Stumps}

\section{Description of Decision Stumps}

\section{Performance of Decision Stumps}



\section{Decision stumps and CART}

Both of the unboosted learning algorithms used in this thesis are
tree-based algorithms.  A tree-based algorithm splits the input space
$\calI$ into a number of disjoint regions.  Each of these regions may
again be split, and the algorithm continues in this recursive manner
until a termination condition is met.  Thus a tree structure is built
up.

\subsection{Decision stumps}

The decision stumps algorithm divides the input space into exactly two
regions, each of which is assigned a category.  The tree created is
very small, with only two nodes (hence the name ``decision
\emph{stumps}''.

\subsubsection{Classification}

An example of a decision boundary generated by a decision stumps
classifier is shown in figu


\subsubsection{Training}



The algorithm for the training of a decision stump is given below

\subsection{CART}

\subsubsection{Classification}

The CART algorithm splits its input space into 


  This is then repeated recursively to refine
the classification.

Both 

\subsubsection{Training}







