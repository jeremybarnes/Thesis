% cdrom.tex
% Jeremy Barnes, 16/10/1999
% $Id$

\chapter{Contents of the attached CD-ROM}
\label{appendix:cdrom}

The attached disk is a standard 650MB cd-recordable that should be
readable in any CD-ROM drive.  It contains a Microsoft Joliet file
system, readable by the Windows 95 and Linux%
\footnote{With the appropriate kernel options installed.}
operating systems.

The root directory of the CD-ROM contains the following subdirectories
and files:
%
\begin{itemize}
\item	{\tt cvsroot} contains the CVS repository for all software
	developed.
\item	{\tt datasets} contains \MATLAB\ versions of all datasets
	used.  By setting the {\tt DATASET\_SAVE\_PATH} global variable
	in \MATLAB\ to point to this directory, the {\tt dataset}
	class will be able to find these datasets.
\item	{\tt data} contains the test and summary files for all of the
	tests that were completed (the raw data was in the order of
	2.5 Gigabytes compressed and could not be included).
\item	{\tt thesis} contains a checked-out version of the {\tt
	thesis} CVS module (this document).
\item	{\tt matlab} contains a checked-out version of the {\tt
	matlab} CVS module.  By adding this directory and all of its
	subdirectories that do not begin with an @ character to the
	\MATLAB\ search path, it should be possible to execute the
	code and use the \MATLAB\ toolbox.
\item	{\tt license.txt} describes the conditions under which the
	information on the CD-ROM may be used, modified and
	distributed.
\item	{\tt matlab\_install.txt} describes in detail how to install
	and build the \MATLAB\ toolbox.  It also describes the
	necessary environment (both external and internal to \MATLAB)
	that is required to use the software.
\item	{\tt matlab\_tutorial.txt} is an introduction to the \MATLAB\
	toolbox, describing its structure and capabilities.
\item	{\tt accessing\_tests.txt} describes how to view the test
	result summaries included in the {\tt data} subdirectory.
\end{itemize}

