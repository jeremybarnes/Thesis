

There are two directions in which this project can move.  The
first to complete exhaustive testing of the $p$-boosting algorithm
in its current form, 
concentrating on \emph{measuring} and possibly \emph{explaining}, but
not \emph{improving} its performance.  This would involve running many
trials unver varying conditions, and using statistical techniques to
analyse performance.  It is a narrowly focused but deep option.

The second possibility is to concentrate on the theoretical
formulation of the algorithm, working on incorporating the $p$-convex
combination into the gradient descent framework.  This would have the
unfortunate side effect of leaving significantly less time for
testing.  It is a broad but shallow investigation.

The available evidence is not sufficient to make this decision.  Some
time needs to be invested in determining whether or not the original
$p$-boosting algorithm is worth investigating.  At a minimum, this
will involve running a few carefully designed tests from which
conclusions can be drawn.

On the available evidence (which is admittedly scant), the second
option seems preferable.  This is for the following reasons:

\begin{itemize}

\item	The $p$ parameter does not currently seem to have much effect
	on the performance of the algorithm (although it is unclear
	under what conditions this is true).

\item	The original $p$-boosting algorithm is something of a hack,
	and is probably not actually performing a sensible gradient
	descent.

\item	The outcome of the second option (an improved algorithm with a
	theoretical justification) is preferable to the first (a
	thorough investigation of the performance of an algorithm).

\item	A structured theoretical approach seems a more elegant way to
	tackle a problem than throwing a lot of computer time at it.

\end{itemize}

The intended plan of action can be summarised as follows:

\begin{enumerate}

\item	Perform theoretical verification of the 


\item	Perform carefully designed tests to determine if the original
	$p$-boosting algorithm has significantly improved performance
	over the boosting algorithm in any circumstance.

\item	Make a decision on which direction to take, contingent on

\end{enumerate}

Thus, the 



\section{Development of a suitable algorithm}

\section{Analysis of the optimal $p$-convex boosting algorithm}

\subsection{Practical analysis}

\subsection{Theoretical analysis}

\section{Production of thesis}

The final outcome of this project is a thesis to be submitted for
marking.  This section describes those tasks which must be performed
in order to complete a thesis.


\footnote{(\ref{eqn:theory:cost
function}) is a crude approximation to the threshold function 
\[ g(x) = \left\{ \begin{array}{ll}
1 \qquad \qquad & \mbox{if $x<0$} \\
0		& \mbox{otherwise} \\
\end{array} \right. \]
that is easy to work with theoretically.}



 \footnote{To first order in $b$, $C(F+bf) = C(F) +
b \ip{\nabla C(F)}{f}$}.



\providecommand{\emp}{\mathrm{emp}}
\providecommand{\calF}{\ensuremath{\mathcal{F}}}
\providecommand{\fat}{\ensuremath{\mathrm{fat}}}
\providecommand{\sign}{\ensuremath{\mathrm{sign}}}
\providecommand{\cop}{\ensuremath{\mathrm{co}_p}}
\providecommand{\co}{\ensuremath{\mathrm{co}}}
\providecommand{\bfx}{\ensuremath{\mathbf{x}}}
\providecommand{\bfy}{\ensuremath{\mathbf{y}}}
\providecommand{\bfyh}{\ensuremath{\hat{\mathbf{y}}}}
\providecommand{\calO}{\ensuremath{\mathcal{O}}}
\providecommand{\calI}{\ensuremath{\mathcal{I}}}
\providecommand{\calH}{\ensuremath{\mathcal{H}}}


\chapter{Details of the boosting algorithm}
\label{chapter:boosting details}

This chapter provides an explicit description of the boosting
algorithm as a complement to the implicit description in chpter
\ref{chapter:theory}.

\par

\noindent{\bf Input:} $l$ examples $(\bfx_1, \bfy_1), \ldots, (\bfx_l,
\bfy_l)$
\par
\noindent{\bf Initialisation:} $w_i|_{t=1} = 1/l$ for $i=1 \ldots l$
\par
\noindent {\bf Do for} $t=1 \ldots T$:


\begin{enumerate}

\item	Train weak learner on the weighted sample set 
	$(\bfx_1, \bfy_1, w_1) \ldots (\bfx_l, \bfy_l, w_l)$
	and obtain weak learner $f_t(\cdot) : \calI \mapsto \{\pm 1\}$

\item	Calculate the training error $\epsilon_t$ of $f_t(\cdot)$:
	%
	\begin{equation}
	\epsilon_t = \sum_{j=1}^l w_j |_t 
	I \left( f_t(\bfx_j) \neq \bfy_j\right)
	\end{equation}

	If $\epsilon_t = 0$ (a single weak learner can correctly learn
	the relationship) or $\epsilon_t \geq \phi - \Delta$ (where
	$\Delta$ is a small positive constant; thus the weak
	learner is performing as badly as random guessing) then abort
	the training process.

\item	Calculate our classifier weight $b_t$:
	\begin{equation}
	b_t = \log \frac{\epsilon_t}{1 - \epsilon_t}
	\end{equation}

\item	Update the weights $w_i$:
	%
	\begin{equation}
	w_i|_{t+1} = \left\{
	\begin{array}{cl}
		w_i|_t / Z_t	&	\qquad \qquad \mbox{if
		$f_t(x_i) = y_i$} \\
		w_i|_t / Z_t \exp \left\{ -b_t \right\} & \qquad \qquad
		\mbox{otherwise} \\
	\end{array} \right.
	\end{equation}

	where $Z_t$ is a normalisation constant, such that
	$\sum_{i=1}^{l} w_i|_{t+1} = 1$
\end{enumerate}

\par
\noindent {\bf Output:} 
\begin{equation}
F'(\bfx) = \sign \left\{ \sum_{i=1}^T w_i f_i(\bfx) \right\}
\tag{\ref{eqn:output}}
\end{equation}


\chapter{Background theory}

This chapter contains a brief summary of the main theoretical results
necessary to understand the rest of the report.  Familiarity with the
field of statistical learning theory is assumed.

Which works does it follow?

%Note that none of these results are original work.  They are merely
%a summary of other author's work.  References are given for the
%source of the results.


\section{Notation}

We begin with some notation.  A classifier is characterised by a
function $f(\cdot) : \mathcal{I} \mapsto \mathcal {O}$.  \calI\
(the \emph{input set}) is the domain of the classifier, and
$\calI \subseteq \mathbb{R}^n$.  For a classifier, \calO\ (the \emph{output set}) is a finite subset of the real numbers.  We will restrict our attention to
$\calI = [0,1] \times [0,1]$ (two dimensional) and $\calO = \{-1,1\}$
(binary) for simplicity.\footnote{Generalisation to other situations is not
usually difficult (see for example \cite{Freund96} and
\cite{Cherkassky98}).}

Samples are notated $\{\bfx, \bfy\}$ where $\bfx \in \calI$ is the
attribute vector and $\bfy \in \calO$ the class.  The classification
process (performed by a learning algorithm when given a sequence of
attributes $\bfx_1, \ldots, \bfx_l$ produces classes \bfyh\ (where
$\bfyh = f(\bfx)$).  The hat distinguishes the \emph{estimates} made
by the learning algorithm from the ``true'' samples.

The ``weak'' learning algorithms decision stumps and CART are
described in appendix \ref{chapter:weak learners}.  Weak learning
algorithms are notated in lowercase $f_i(\cdot)$, where $i$ is an
index and each $f_i$ is an element of \calF, which is the set of all
appropriate instances of a learning algorithm (for example, all
decision stump classifiers that operate in two dimensions on a binary
problem). 

\section{Boosting}

Boosting is a learning algorithm that uses a linear combination of
many ``weak'' learning algorithms to generate a combined hypothesis
that is usually significantly better than that of any weak learning algorithm.
Even the simplest of weak algorithms, when boosted, will usually
outperform the best unboosted algorithms.
\footnote{As a result, the focus of recent research effort has shifted from the
development of \emph{weak learning} algorithms (which are all more
or less equivalent when boosted) to the development of better \emph{boosting}
algorithms.}

Boosting builds a strong classifier from a linear combination of
weak classifiers.  The weight of a weak classifier (the
\emph{classifier weight}) depends upon the performance of that
classifier on the training dataset.  Section \ref{sec:classifier
weights} describes these classifier weights in more detail.

Samples in the training dataset are given weights (\emph{sample
weights}) which are modified depending upon how ``hard'' that
datapoint is to classify.  Section \ref{sec:sample weights} describes
these sample weights in more detail.

A more thorough description of boosting is available in appendix
\ref{chapter:boosting details} and in \cite{Freund96}.


\subsection{Classifier weights}
\label{sec:classifier weights}

The boosting algorithm combines a number of ``weak'' classifiers
$f_1(\cdot), \ldots, f_n(\cdot)$ in a linear combination to produce a
``stronger'' classifier $F(\cdot) = \sum_{i=1}^{n} b_i f_i(\cdot)$.
The coefficients $b_i$ are subject to the condition that they produce a
\emph{convex} combination:
%
\begin{equation}
\label{eqn:p=1}
\|b\|_1 = \sum_{i=1}^{t} b_i = 1
\end{equation}
%

The training process of boosting may be conveniently divided into
iterations.  On iteration $t$, one
weak learner $f_t(\cdot)$ is added to the linear combination.
The coefficient $b_t$ is calculated from the training error $\epsilon_t$ 
%
\begin{equation}
\epsilon_t = \sum_{j=1}^l w_j |_t 
I \left( f_t(\bfx_j) \neq \bfy_j\right)
\label{eqn:training error}
\end{equation}
%
where $l$ is the number of training samples.  (The notation $w_j|_t$
means the value of $w_j$ at iteration $t$.  The role of these weights
is discussed in section \ref{sec:sample weights} below).

The coefficient $b_t$ is generated from
%
\begin{equation}
b_t = \log \frac{\epsilon_t}{1 - \epsilon_t}
\label{eqn:bt}
\end{equation}
%
When $\epsilon_t = 1/2$, the classifier only does as
well as random guessing; thus it has a weight of zero.  As
$\epsilon_t$ approaches zero, the weight increases, and
the more influence this classifier has on $F$.
\footnote{Training is halted if the classification error reaches
zero.}  A plot of $b_t$ against $\epsilon_t$ is given in appendix
\ref{chapter:proposal}.


\subsection{Sample weights}
\label{sec:sample weights}

Each sample in the training set has a corresponding weight $w_j$ which
describes the difficulty of that sample.  These weights are
initialised to $w_i|_{t=1} = 1/l$, and are updated each iteration
according to
%
\begin{equation}
w_i|_{t+1} = \left\{
\begin{array}{ll}
	\frac{w_i|_t}{Z_t}	&	\qquad \qquad \mbox{if
	$f_t(x_i) = y_i$} \\
	\frac{w_i|_t}{Z_t} \exp \left\{ -b_t \right\} & \qquad \qquad
	\mbox{otherwise} \\
\end{array} \right.
\label{eqn:sample weight update}
\end{equation}
%
where $Z_t$ is a normalisation constant, such that
$\sum{i=1}{l} w_i|_{t+1} = 1$.  This function has the following effect
on the weights:

\begin{itemize}

\item	Training samples which are misclassified often, or are one of 
	few points to get misclassified, increase their proportion of
	the weight (are identified as ``hard samples'';

\item	Other samples decrease their proportion of the weight, and are
	identified as ``easy'' samples.

\end{itemize}

The overall effect is for those samples near the decision boundary to
increase in weight, and those far from it to decrease.  This forces
the algorithm to concentrate on the hard samples, and is one reason
why it works well.  Unfortunately, those samples corrupted by noise
are usually hard also, and concentrating on those is not desirable.


\subsection{Classification}
The final output $F'(\bfx)$ of the boosting algorithm is determined by
the sign of $F(\bfx)$:
\begin{equation}
F'(\bfx) = \sign \left\{ F(\bfx) \right\} = \sign \left\{ \sum_{i=1}^T
w_i f_i(\bfx) \right\}
\label{eqn:output}
\end{equation}

\section{Margins}

The margin of a classifier $F(\cdot) : \calI \mapsto \{\pm 1\}$ on a
sample $(\bfx, y)$ is defined as
%
\begin{equation}
m_F(\bfx, y) \doteq y F(\bfx)
\end{equation}
%
The margin is a measure of how confidently the classifier was able to
predict the correct result.  It will be positive if the classification
was correct and negative otherwise.

\section{Boosting as gradient descent}
\label{sec:theory:gradient descent}

Recent work has
shown boosting to be an implementation of gradient descent in an inner
product space \cite{Mason99}\footnote{This discussion follows
\cite{Mason99} rather closely.}.
Noting that an inner product space requires both a universe
$\mathcal{X}$ and an inner product operator \ip{\cdot}{\cdot}, we
define
%
\begin{equation}
\calX = 
\mathrm{co} (\calF) \doteq
 \bigcup _{n \in \mathbb{N}}
\left\{
 \sum_{i=1}^{n}
 \alpha _i
f_i : f_1, \ldots, f_n \in \calF,
 \alpha _1, \ldots, \alpha _n \in \mathbb{R},
 \sum_{i=1}^{n} | a_i | = 1
\right\} \cup \emptyset
\end{equation}
%
(which is the convex hull of \calF), and
%
\begin{equation}
\ip{F}{G} = \frac{1}{l} \sum_{i=1}^{l} F(\bfx_i)G(\bfx_i)
\end{equation}
%
We also choose the cost function\footnote{(\ref{eqn:theory:cost
function}) is a crude approximation to the threshold function 
\[ g(x) = \left\{ \begin{array}{ll}
1 \qquad \qquad & \mbox{if $x<0$} \\
0		& \mbox{otherwise} \\
\end{array} \right. \]
that is easy to work with theoretically.}
%
\begin{equation}
C(F) = \frac{1}{l} \sum_{i=1}^{l} \exp
\left\{ -y_i F(\bfx_i) \right\}
\label{eqn:theory:cost function}
\end{equation}

At iteration $t$ of gradient descent, we wish to choose a function $f_t$ and
a weight $b_t$ such that $C(F + b_t f_t)$ decreases (the choice of
$f_t$ is made indirectly through the choice of the sample weights
$w|_t$).  In other words, we are asking for sample weights $w|_t$ that
choose a \emph{direction} $f_t \in \calF$ such that $C(F + b_t f_t)$
decreases as fast as possible.  This direction is the negative of the
functional derivative of $C$:
%
\begin{equation}
f = -\nabla C(F)(x) = \left. \frac{\partial C(F + \alpha
1_x)}{\partial \alpha} \right|_{\alpha = 0}
\end{equation}
%
where $1_x$ is the indicator function of $x$.  Since the
optimal $f_t$ will not necessarily be in $\calF$, we instead choose the
$f \in \calF$ with the greatest inner product $\ip{-\nabla
C(F)}{f}$ \footnote{To first order in $b$, $C(F+bf) = C(F) +
b \ip{\nabla C(F)}{f}$}.  Solving for $w_i|_t$, $i=1 \ldots l$, we
obtain (\ref{eqn:sample weight update}).

Having chosen our direction, we now choose a step size $b_t$.
There are many schemes for choosing this step size; boosting uses a
line search for the minimum of the cost functional along this line
%
\begin{equation}
\sum_{i=1}^{l} C(y_i F(\bfx_i) + y_i b_t f_t(\bfx_i)
\end{equation}
%
which (for the choice of cost function (\ref{eqn:theory:cost
function})) has a closed-form solution (\ref{eqn:bt}).  Thus, the
boosting algorithm implements gradient descent.  (Some details, such
as the stopping criteria, have been glossed over; the result still
holds).
